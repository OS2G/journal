\documentclass{article}
\usepackage[utf8]{inputenc}

\usepackage{soul}
\usepackage{color}

\usepackage{setspace} 
\doublespacing

\usepackage[a4paper, margin=0.75in]{geometry}

\usepackage{pgffor}
\usepackage{import}

\usepackage{makeidx}
\makeindex

\usepackage{hyperref}
\hypersetup{
    colorlinks=true, %set true if you want colored links
    linktoc=all,     %set to all if you want both sections and subsections linked
    % linkcolor=blue,  %choose some color if you want links to stand out
}


\setlength{\parindent}{0pt}

\title{OS2G Journal 2022-23}
\date{ }

\newcommand{\president}[0]{Aaron Friesen}
\newcommand{\vicepresident}[0]{Connor Kildare}
\newcommand{\tresurer}[0]{Timothy Gibbons}
\newcommand{\primaryprogrammer}[0]{Erbey Uribe}
\newcommand{\secretary}[0]{Luke Freyhof}

\newcommand\TwoDigits[1]{%
   \ifnum#1<10 0#1\else #1\fi
}

\begin{document}
\maketitle
\newpage
\tableofcontents
\newpage
\section{General Notes}
\subsection{Missing Notes}
This journal contain missing notes from the beginning of the Fall 2022 semester. This is because the journal was started Fri Sep, 23rd, 2022.

\subsection{FOSS Jam}
FOSS (Free and Open Source Software) Jam is an event where members can start on or contribute to projects. 
Members found that university doesn't provide enough tools or resources to help with starting and contributing to projects by the time they take Senior Design.
Some of the goal we have in mind is to provide a platform where students in engineering could learn how to start a project, collaborate on other projects, continue with that project after the event, and providing back to the community.
What makes FOSS Jam different from a typical Hack-a-thon, is that contributing to other projects is allowed.

\newpage

\foreach \YYYY in {2022,...,2023}{%
    \foreach \MM in {1,2,...,12}{%
        \foreach \DD in {1,2,...,31}{%
            \edef\FileName{\YYYY\TwoDigits{\MM}\TwoDigits{\DD}}
            \IfFileExists{entry/\FileName.tex} {%
                % \FileName
                \import{entry/}{\FileName.tex}%
                \newpage
            }{%
                    % files does not exist, so nothing to do
            }%
        }%
    }%
}%
\end{document}
